\begin{answer}
\ 
\begin{subanswer}
    \ \\
    由$Bayes$公式得
    \begin{align}
    \bayes{H}{E_1}{i}{n}
    \end{align}
    其中
    \begin{align*}
    &\sum\limits_{i=1}^{n}P(H_i)P(E_1|H_i) \\ 
    %=& P(H_1)P(E_1|H_1) \cdot P(H_2)P(E_1|H_2) \cdot P(H_3)P(E_1|H_3) \\
    =&\sum\limits_{i=1}^{3} P(H_i)P(E_1|H_i) \\
    =& (0.4 \times 0.5) + (0.3 \times 0.3) + (0.3 \times 0.5) \\
    =& 0.44
    \end{align*}
    \begin{align}
    \bayes{H}{E_1}{1}{3} \notag\\
    =&\frac{0.4 \times 0.5}{0.44} \notag\\
    \approx&0.45 > P(H_1)
    \end{align}
    \begin{align}
    \bayes{H}{E_1}{2}{3} \notag\\
    =&\frac{0.3 \times 0.3}{0.44} \notag\\
    \approx&0.20 < P(H_2)
    \end{align}
    \begin{align}
    \bayes{H}{E_1}{3}{3} \notag\\
    =&\frac{0.3 \times 0.5}{0.44} \notag\\
    \approx&0.34 > P(H_3)
    \end{align}
    可见由于证据$E_1$的出现,$H_1,H_3$成立的可能性有所增加,而$H_2$成立的可能性有所下降。
\end{subanswer}
\begin{subanswer}
    \ \\
    由$Bayes$公式得
    \begin{align}
    \Bayest
    \end{align}
    其中
    \begin{align*}
    &\sum\limits_{i=1}^{n}\big(P(H_i)\prod\limits_{j=1}^{m}P(E_j|H_i)\big) \\
    =&\sum\limits_{i=1}^{3}P(H_i)P(E_1|H_i)P(E_2|H_i) \\
    =&\big(0.4 \times (0.5 \times 0.7)\big)+ \\
    &\big(0.3 \times (0.3 \times 0.9)\big)+ \\
    &\big(0.7 \times (0.5 \times 0.1)\big) \\
    =&0.256
    \end{align*}
    \begin{align}
    \Bayes{1} \notag\\
    =&\frac{0.4 \times (0.5 \times 0.7)}{0.256} \notag\\
    \approx&0.54>P(H_1)
    \end{align}
    \begin{align}
    \Bayes{2} \notag\\
    =&\frac{0.3 \times (0.3 \times 0.9)}{0.256} \notag\\
    \approx&0.31>P(H_2)
    \end{align}
    \begin{align}
    \Bayes{3} \notag\\
    =&\frac{0.7 \times (0.5 \times 0.1)}{0.256} \notag\\
    \approx&0.13<P(H_3)
    \end{align}
    可见由于证据$E_1,H_2$的出现,$H_1$成立的可能性有所增加,$H_2$成立的可能性略有增加,$H_3$成立的可能性有所下降。
\end{subanswer}
\end{answer}
