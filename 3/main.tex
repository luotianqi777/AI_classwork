\documentclass{article}

\usepackage{ctex}
\usepackage{amsmath}
\usepackage{xifthen}
\usepackage[a4paper,left=1.5cm,right=1.5cm,top=2cm,bottom=2.5cm]{geometry}
\newtheorem{answer}{Answer}
\newtheorem{subanswer}{answer}[answer]
\newcommand{\bayes}[4]{
	P(#1_#3|#2)=&
	\frac{P(#1_#3)P(#2|#1_#3)}
	{\sum\limits_{j=1}^{#4} P(#1_j)P(#2|#1_j)}
}
\newcommand{\Bayest}{
	P(H_i|E_1E_2\dots E_m)=
	\frac{P(H_i)\prod\limits_{j=1}^{m}P(E_j|H_i)}
	{\sum\limits_{j=1}^{n}\big(P(H_j)\prod\limits_{k=1}^{m}P(E_k|H_j)\big)}
}
\newcommand{\Bayes}[1]{
	P(H_#1|E_1E_2)=&
	\frac{P(E_1|H_#1)P(E_2|H_#1)P(H_#1)}
	{\sum\limits_{i=1}^{3}P(H_i)P(E_1|H_i)P(E_2|H_i)}
}

\title{人工智能作业}
\author{信息与计算科学1601\\ 骆天奇\\ 2016254060407}
\date{\today}
%\date{12.4}
\twocolumn

\begin{document}
%	\maketitle
	\paragraph{信息与计算科学1601 骆天奇 2016254060407}
	\documentclass{article}
\usepackage{ctex}
\usepackage{verbatim}
\usepackage{listings}
\usepackage[left=1in,right=1in,top=0.7in,bottom=0.7in]{geometry}
\twocolumn

\usepackage{tikz}
\usetikzlibrary{shapes,arrows}
\tikzstyle{r} = [rectangle, minimum width=2cm, minimum height=1cm, text centered, draw=black]

\title{人工智能作业}
\author{骆天奇\\2016254060407}
%\date{19.11.6}
\date{\today}

\begin{document}
	\maketitle
	\section{semantic networks}
	\begin{tikzpicture}[node distance = 1.5cm]
		\node[r](nellie){Nellie};
		\node[r,below of=nellie](ele){Elephant};
		\node[r,left of=nellie, xshift=-1cm](apple){Like Apple};
		\node[r,left of=ele, xshift=-1cm](africa){In Africa};
		\node[r,right of=ele, xshift=1cm](big){Big};
		\node[r,below of=ele](mammals){Mammals};
		\node[r,below of=mammals](animals){Animals};
		\node[r,left of=animals, xshift=-1cm](head){Have Head};
		\node[r,right of=animals, xshift=1cm](reptiles){Reptiles};
		
		\draw[->](nellie)--(ele);
		\draw[->](nellie)--(apple);
		\draw[->](ele)--(mammals);
		\draw[->](ele)--(africa);
		\draw[->](ele)--(big);
		\draw[->](mammals)--(animals);
		\draw[->](reptiles)--(animals);
		\draw[->](animals)--(head);
	\end{tikzpicture}
	\section{汉诺塔结果}
		\verbatiminput{code/out.txt}
	\section{汉诺塔代码}
		\verbatiminput{code/hannuota.py}
	
\end{document}

%	\begin{answer}
	\ \\
	不会
\end{answer}
\end{document}